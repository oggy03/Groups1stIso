% In this file you should put the actual content of the blueprint.
% It will be used both by the web and the print version.
% It should *not* include the \begin{document}
%
% If you want to split the blueprint content into several files then
% the current file can be a simple sequence of \input. Otherwise It
% can start with a \section or \chapter for instance.



% Preamble: Load necessary packages

\maketitle

This document outlines the proof of the First Isomorphism Theorem, a fundamental result in group theory. The structure follows a logical progression from basic definitions to the final theorem, mirroring the formalization presented in the provided Lean code.

\section{Foundational Concepts}

We begin by defining the core algebraic structures and concepts.

\begin{definition}[Group]
A \emph{group} is a set $G$ equipped with a binary operation $*: G \times G \to G$, an identity element $1 \in G$, and an inverse map $^{-1}: G \to G$, satisfying the following axioms for all $a, b, c \in G$:
\begin{enumerate}
    \item \textbf{Associativity:} $a * (b * c) = (a * b) * c$
    \item \textbf{Identity Element:} $a * 1 = a$ and $1 * a = a$
    \item \textbf{Inverse Element:} $a * a^{-1} = 1$ and $a^{-1} * a = 1$
\end{enumerate}
From these axioms, standard properties such as the uniqueness of the identity and inverses, the cancellation laws, and the identity $(a*b)^{-1} = b^{-1}*a^{-1}$ can be derived and will be used throughout this proof.
\end{definition}

\begin{definition}[Subgroup]
A subset $H$ of a group $G$ is a \emph{subgroup} if it forms a group under the operation inherited from $G$. This is true if and only if:
\begin{enumerate}
    \item \textbf{Identity:} $1 \in H$
    \item \textbf{Closure under Multiplication:} For all $a, b \in H$, $a * b \in H$
    \item \textbf{Closure under Inverses:} For all $a \in H$, $a^{-1} \in H$
\end{enumerate}
\end{definition}

\begin{definition}[Normal Subgroup]
A subgroup $N$ of a group $G$ is a \emph{normal subgroup}, denoted $N \trianglelefteq G$, if for every $g \in G$ and $n \in N$, the conjugate element $gng^{-1}$ is also in $N$.
\[
\forall g \in G, \forall n \in N \quad gng^{-1} \in N
\]
\end{definition}

\begin{definition}[Group Homomorphism]
A \emph{group homomorphism} is a map $f: G \to H$ between two groups $(G, *_G)$ and $(H, *_H)$ that preserves the group operation:
\[
\forall a, b \in G \quad f(a *_G b) = f(a) *_H f(b)
\]
From this property, it follows that $f(1_G) = 1_H$ and $f(a^{-1}) = (f(a))^{-1}$ for all $a \in G$.
\end{definition}

\begin{definition}[Kernel and Image]
Given a group homomorphism $f: G \to H$:
\begin{itemize}
    \item The \textbf{kernel} of $f$, denoted $\ker(f)$, is the set of elements in $G$ that map to the identity element in $H$.
    \[ \ker(f) := \{ g \in G \mid f(g) = 1_H \} \]
    \item The \textbf{image} of $f$, denoted $\im(f)$, is the set of elements in $H$ that are the image of some element in $G$.
    \[ \im(f) := \{ h \in H \mid \exists g \in G, f(g) = h \} \]
\end{itemize}
\end{definition}

\section{Constructing the Quotient Group}

A key step in the theorem is the construction of a new group from a group $G$ and one of its normal subgroups $N$. This new group is called the quotient group, denoted $G/N$.

\subsection*{Step 2.1: An Equivalence Relation}
Let $N$ be a normal subgroup of $G$. We define a relation $\sim$ on $G$ by:
\[
a \sim b \quad \iff \quad a^{-1}b \in N
\]
We now prove this is an equivalence relation.
\begin{proof}
\begin{itemize}
    \item \textbf{Reflexivity:} For any $a \in G$, $a^{-1}a = 1$. Since $N$ is a subgroup, $1 \in N$. Thus, $a \sim a$.
    \item \textbf{Symmetry:} Suppose $a \sim b$, so $a^{-1}b \in N$. Since $N$ is closed under inverses, $(a^{-1}b)^{-1} \in N$. As $(a^{-1}b)^{-1} = b^{-1}a$, we have $b^{-1}a \in N$, which means $b \sim a$.
    \item \textbf{Transitivity:} Suppose $a \sim b$ and $b \sim c$. Then $a^{-1}b \in N$ and $b^{-1}c \in N$. Since $N$ is closed under multiplication, $(a^{-1}b)(b^{-1}c) \in N$. This simplifies to $a^{-1}(bb^{-1})c = a^{-1}c \in N$, so $a \sim c$.
\end{itemize}
\end{proof}

\subsection*{Step 2.2: The Set of Cosets G/N}
The equivalence classes of this relation are precisely the left cosets of $N$ in $G$. The equivalence class of an element $a \in G$ is:
\[
[a] = \{ g \in G \mid a \sim g \} = \{ g \in G \mid a^{-1}g \in N \} = \{ g \in G \mid g = an \text{ for some } n \in N \} =: aN
\]
The set of all such equivalence classes (cosets) is denoted $G/N$.

\subsection*{Step 2.3: A Well-Defined Group Operation on G/N}
We define a binary operation on $G/N$ as follows:
\[
(aN) * (bN) := (ab)N
\]
For this definition to be valid, we must show that the result does not depend on the choice of representatives $a$ and $b$.
\begin{proof}
Let $aN = a'N$ and $bN = b'N$. This means $a' = an_1$ and $b' = bn_2$ for some $n_1, n_2 \in N$. We must show that $(ab)N = (a'b')N$, which is equivalent to showing $(ab)^{-1}(a'b') \in N$.
\begin{align*}
(ab)^{-1}(a'b') &= (b^{-1}a^{-1})(an_1bn_2) \\
&= b^{-1}(a^{-1}a)n_1bn_2 \\
&= b^{-1}n_1bn_2
\end{align*}
Because $N$ is a normal subgroup, $b^{-1}n_1b \in N$. Let $n_3 = b^{-1}n_1b$. The expression becomes $n_3n_2$. Since $N$ is a subgroup, it is closed under multiplication, so $n_3n_2 \in N$. Thus, the operation is well-defined.
\end{proof}

\subsection*{Step 2.4: Verifying the Group Axioms for G/N}
The set $G/N$ with this operation forms a group:
\begin{itemize}
    \item \textbf{Associativity:} Follows directly from associativity in $G$.
    \[((aN)(bN))(cN) = (ab)N(cN) = ((ab)c)N = (a(bc))N = aN(bc)N = aN((bN)(cN))\]
    \item \textbf{Identity:} The identity element is the coset $1N = N$, since $(aN)(1N) = (a1)N = aN$.
    \item \textbf{Inverse:} The inverse of the coset $aN$ is $a^{-1}N$, since $(aN)(a^{-1}N) = (aa^{-1})N = 1N = N$.
\end{itemize}

\section{The First Isomorphism Theorem}

We now have all the components to state and prove the theorem.

\begin{lemma}
For any group homomorphism $f: G \to H$, $\ker(f)$ is a normal subgroup of $G$.
\end{lemma}
\begin{proof}
We first show $\ker(f)$ is a subgroup.
\begin{itemize}
    \item $f(1_G) = 1_H$, so $1_G \in \ker(f)$.
    \item If $a, b \in \ker(f)$, then $f(a)=1_H$ and $f(b)=1_H$. So $f(ab) = f(a)f(b) = 1_H \cdot 1_H = 1_H$, meaning $ab \in \ker(f)$.
    \item If $a \in \ker(f)$, then $f(a^{-1}) = (f(a))^{-1} = (1_H)^{-1} = 1_H$, meaning $a^{-1} \in \ker(f)$.
\end{itemize}
Next, we show it is normal. Let $k \in \ker(f)$ and $g \in G$. We check if $gkg^{-1} \in \ker(f)$.
\[
f(gkg^{-1}) = f(g)f(k)f(g^{-1}) = f(g) \cdot 1_H \cdot (f(g))^{-1} = f(g)(f(g))^{-1} = 1_H
\]
Therefore, $gkg^{-1} \in \ker(f)$, so $\ker(f)$ is a normal subgroup of $G$.
\end{proof}

\begin{lemma}
For any group homomorphism $f: G \to H$, $\im(f)$ is a subgroup of $H$.
\end{lemma}
\begin{proof}
\begin{itemize}
    \item $1_H = f(1_G)$, so $1_H \in \im(f)$.
    \item If $h_1, h_2 \in \im(f)$, then $h_1 = f(g_1)$ and $h_2 = f(g_2)$ for some $g_1, g_2 \in G$. Then $h_1h_2 = f(g_1)f(g_2) = f(g_1g_2)$, so $h_1h_2 \in \im(f)$.
    \item If $h \in \im(f)$, then $h=f(g)$. Then $h^{-1} = (f(g))^{-1} = f(g^{-1})$, so $h^{-1} \in \im(f)$.
\end{itemize}
Thus, $\im(f)$ is a subgroup of $H$.
\end{proof}

\begin{theorem}[First Isomorphism Theorem]
Let $f: G \to H$ be a group homomorphism. Then the quotient group $G/\ker(f)$ is isomorphic to the image group $\im(f)$.
\[ G/\ker(f) \cong \im(f) \]
\end{theorem}
\begin{proof}
Let $K = \ker(f)$. Since $K$ is a normal subgroup of $G$, the quotient group $G/K$ exists. We define a map $\phi: G/K \to \im(f)$ by:
\[
\phi(gK) = f(g)
\]
We must prove that $\phi$ is an isomorphism by showing it is well-defined, a homomorphism, injective, and surjective.
\begin{enumerate}
    \item \textbf{$\phi$ is Well-Defined:} Suppose $g_1K = g_2K$. We need to show $\phi(g_1K) = \phi(g_2K)$, i.e., $f(g_1) = f(g_2)$. The condition $g_1K=g_2K$ is equivalent to $g_2^{-1}g_1 \in K$. By the definition of the kernel, $f(g_2^{-1}g_1) = 1_H$. Using the homomorphism property, $f(g_2^{-1})f(g_1) = (f(g_2))^{-1}f(g_1) = 1_H$. Multiplying by $f(g_2)$ on the left gives $f(g_1) = f(g_2)$. Thus, $\phi$ is well-defined.

    \item \textbf{$\phi$ is a Homomorphism:} We check if $\phi((g_1K)(g_2K)) = \phi(g_1K)\phi(g_2K)$.
    \[ \phi((g_1K)(g_2K)) = \phi((g_1g_2)K) = f(g_1g_2) \]
    \[ \phi(g_1K)\phi(g_2K) = f(g_1)f(g_2) \]
    Since $f$ is a homomorphism, $f(g_1g_2) = f(g_1)f(g_2)$. Thus, $\phi$ is a homomorphism.

    \item \textbf{$\phi$ is Injective (One-to-one):} We show that the kernel of $\phi$ is the trivial subgroup of $G/K$, which consists only of the identity element $K$.
    \[ \ker(\phi) = \{ gK \in G/K \mid \phi(gK) = 1_H \} \]
    The condition $\phi(gK) = f(g) = 1_H$ is true if and only if $g \in K$.
    So, the kernel of $\phi$ is the set $\{gK \mid g \in K\}$, which is precisely the single element $K$. Since $\ker(\phi)$ is trivial, $\phi$ is injective.

    \item \textbf{$\phi$ is Surjective (Onto):} Let $h$ be an arbitrary element of $\im(f)$. By the definition of image, there exists some $g \in G$ such that $f(g) = h$. Consider the coset $gK \in G/K$. Then $\phi(gK) = f(g) = h$. Thus, every element in $\im(f)$ has a preimage in $G/K$, so $\phi$ is surjective.
\end{enumerate}
Since $\phi$ is a well-defined, bijective homomorphism, it is a group isomorphism. This completes the proof.
\end{proof}

