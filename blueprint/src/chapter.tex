
\chapter{Algebraic Preliminaries}

\section{Basic structures}

\begin{definition}[Group]\label{def:mygroup}
\lean{OurGroup.MyGroup}
A \emph{group} is a type $G$ equipped with a multiplication
$(\cdot)$, identity element $1$, and inverse $(-)^{-1}$ satisfying
the usual axioms:
\[
(a\cdot b)\cdot c = a\cdot(b\cdot c),
\quad
1\cdot a = a = a\cdot1,
\quad
a^{-1}\!\cdot a = 1 = a\cdot a^{-1}.
\]
\end{definition}

\begin{definition}[Abelian group]\label{def:abegroup}
\lean{OurGroup.AbeGroup}
A group $(G,\cdot)$ is \emph{abelian} if $a\cdot b = b\cdot a$ for
all $a,b\in G$.
\uses{def:mygroup}
\end{definition}

\begin{definition}[Subgroup]\label{def:mysubgroup}
\lean{OurGroup.MySubgroup}
A subset $H\subseteq G$ is a \emph{subgroup} if it contains the
identity, is closed under $\cdot$, and closed under inversion.
\uses{def:mygroup}
\end{definition}

\begin{definition}[Normal subgroup]\label{def:normalsubgroup}
\lean{OurGroup.NormalSubgroup}
A subgroup $H\subseteq G$ is \emph{normal} if $g h g^{-1}\in H$ for
all $g\in G$ and $h\in H$.
\uses{mysubgroup}
\end{definition}

\begin{definition}[Cosets and the trivial subgroup]\label{def:cosets}
\lean{OurGroup.LeftCoset,OurGroup.RightCoset,OurGroup.Trivial}
For $H\le G$ and $g\in G$ the \emph{left coset} $gH$ is
$\{gh\mid h\in H\}$, the \emph{right coset} $Hg$ is
$\{h g\mid h\in H\}$, and the \emph{trivial subgroup} is
$\{1\}$.
\uses{def:mysubgroup}
\end{definition}


%%%%%%%%%%%%%%%%%%%%%%%%%%%%%%%%%%%%%%%%%%%%%%%%%%%%%%%%%%%%%%%%%%%%%%%%%%%%%%%%%%
\section{Elementary lemmas}
%%%%%%%%%%%%%%%%%%%%%%%%%%%%%%%%%%%%%%%%%%%%%%%%%%%%%%%%%%%%%%%%%%%%%%%%%%%%%%%%%%

\begin{lemma}[Left cancellation]\label{lem:left_cancel}
\lean{OurGroupProp.mul_left_cancel}
If $a b = a c$ in a group, then $b=c$.
\uses{def:mygroup}
\end{lemma}

\begin{lemma}[Inverse of a product]\label{lem:inv_prod}
\lean{OurGroupProp.inv_mul}
For all $a,b\in G$ one has $(a b)^{-1} = b^{-1} a^{-1}$.
\uses{def:mygroup,lem:left_cancel}
\end{lemma}

%% …include any other lemmas you wish the graph to display …


%%%%%%%%%%%%%%%%%%%%%%%%%%%%%%%%%%%%%%%%%%%%%%%%%%%%%%%%%%%%%%%%%%%%%%%%%%%%%%%%%%
\section{Quotient groups}
%%%%%%%%%%%%%%%%%%%%%%%%%%%%%%%%%%%%%%%%%%%%%%%%%%%%%%%%%%%%%%%%%%%%%%%%%%%%%%%%%%

\begin{definition}[Congruence modulo a normal subgroup]\label{def:normeql}
\lean{OurQuotient.NormEquiv}
For $H\trianglelefteq G$ write $a\sim_H b$ when $a b^{-1}\in H$.
\uses{def:normalsubgroup}
\end{definition}

\begin{lemma}[Compatibility with multiplication]\label{lem:norm_mul}
\lean{OurQuotient.NormEquiv_mul}
If $a_1\sim_H a_2$ and $b_1\sim_H b_2$ then
$a_1 b_1\sim_H a_2 b_2$.
\uses{def:normeql,lem:inv_prod}
\end{lemma}

\begin{definition}[Quotient group]\label{def:quotient_group}
\lean{OurQuotient.MyQuotientGroup}
Write $G/H$ for the set of equivalence classes
under $\sim_H$; it inherits a group structure from~$G$.
\uses{def:normeql,lem:norm_mul}
\end{definition}


%%%%%%%%%%%%%%%%%%%%%%%%%%%%%%%%%%%%%%%%%%%%%%%%%%%%%%%%%%%%%%%%%%%%%%%%%%%%%%%%%%
\section{Homomorphisms}
%%%%%%%%%%%%%%%%%%%%%%%%%%%%%%%%%%%%%%%%%%%%%%%%%%%%%%%%%%%%%%%%%%%%%%%%%%%%%%%%%%

\begin{definition}[Group homomorphism]\label{def:grouphom}
\lean{OurGroupHom.MyGroupHom}
A map $f\colon G\to H$ is a homomorphism if
$f(ab)=f(a)f(b)$, $f(1)=1$, and $f(a^{-1})=f(a)^{-1}$.
\uses{def:mygroup}
\end{definition}

\begin{definition}[Kernel and image]\label{def:ker_img}
\lean{OurGroupHom.Myker,OurGroupHom.Myim}
For a homomorphism $f$ the \emph{kernel} is
$\ker f := \{g\mid f(g)=1\}$ and the \emph{image} is
$\mathrm{im}\,f := \{f(g)\mid g\in G\}$.
\uses{def:grouphom,def:triv_sub} %% trival subgroup is in def:cosets
\end{definition}

\begin{lemma}[Kernel is normal]\label{lem:ker_normal}
\lean{OurFirstIso.Myker_is_normal}
$\ker f$ is a normal subgroup of $G$.
\uses{def:ker_img,def:normalsubgroup}
\end{lemma}

\begin{lemma}[Image is a subgroup]\label{lem:img_group}
\lean{OurFirstIso.Myim_is_group}
$\mathrm{im}\,f$ inherits a group structure from $H$.
\uses{def:ker_img}
\end{lemma}


%%%%%%%%%%%%%%%%%%%%%%%%%%%%%%%%%%%%%%%%%%%%%%%%%%%%%%%%%%%%%%%%%%%%%%%%%%%%%%%%%%
\section{The first isomorphism theorem}
%%%%%%%%%%%%%%%%%%%%%%%%%%%%%%%%%%%%%%%%%%%%%%%%%%%%%%%%%%%%%%%%%%%%%%%%%%%%%%%%%%

\begin{theorem}[First Isomorphism Theorem]\label{thm:first_iso}
\lean{OurFirstIso.FirstIsoTheorem}
\uses{def:quotient_group,lem:ker_normal,lem:img_group}
There exists a natural isomorphism
\[
G/\ker f\;\;\cong\;\;\mathrm{im}\,f .
\]
\end{theorem}
